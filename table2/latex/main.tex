\documentclass[12pt,letterpaper]{article}
\usepackage{graphicx}
\usepackage{xcolor,balance}
\usepackage{todonotes}
\usepackage{amsmath}
\usepackage{amssymb}
\usepackage{subcaption}
\usepackage[font=footnotesize]{caption}
%\usepackage{refcheck}

\usepackage{bm} % for \bm

% prevent conflict with IEEE template
% https://tex.stackexchange.com/questions/170772/command-labelindent-already-defined
\let\labelindent\relax
\usepackage{enumitem}

%\usepackage{ulem}

\usepackage{multirow}
\usepackage{multicol}
\usepackage{colortbl}

% needed for \begin{algorithmic}
% https://www.overleaf.com/learn/latex/Algorithms
\usepackage{algorithm}
\usepackage{algpseudocode} 

\def\XAxisColor{red}
\def\YAxisColor{green}
\def\ZAxisColor{blue}
\def\PhantomColor{gray}
\def\LoRColor{cyan}
\def\XAxisText{$x+$}
\def\YAxisText{$y+$}
\def\ZAxisText{$z+$}
\def\EmptyText{}

\def\MyRColor{orange}


%\usepackage{hyperref}
\usepackage[pdftex, pdfstartview={FitV}, pdfpagelayout={TwoColumnLeft},bookmarksopen=true,plainpages = false, colorlinks=true, linkcolor=black, citecolor = black, urlcolor = black,filecolor=black , pagebackref=false,hypertexnames=false, plainpages=false, pdfpagelabels ]{hyperref}

\usepackage{booktabs}
\usepackage[sort,compress]{cite}

% to allow us to type ~
\usepackage{textcomp}

\newcommand{\mytodo}[1]{\textcolor{red}{#1}}
\newcommand{\myquote}[1]{\textcolor{blue}{#1}}
\newcommand{\mycite}{\textcolor{red}{[?]}}

% https://www.overleaf.com/learn/latex/Using_colors_in_LaTeX
% https://matplotlib.org/stable/gallery/color/named_colors.html
\definecolor{tab-blue}{rgb}{0.122, 0.467, 0.706}
\definecolor{tab-orange}{rgb}{1.000, 0.498, 0.055}
\definecolor{tab-green}{rgb}{0.173, 0.627, 0.173}
\definecolor{tab-red}{rgb}{0.839, 0.153, 0.157}
\definecolor{tab-purple}{rgb}{0.580, 0.404, 0.741}
\definecolor{tab-brown}{rgb}{0.549, 0.337, 0.294}
\definecolor{tab-pink}{rgb}{0.890, 0.467, 0.761}
\definecolor{tab-gray}{rgb}{0.498, 0.498, 0.498}
\definecolor{tab-olive}{rgb}{0.737, 0.741, 0.133}
\definecolor{tab-cyan}{rgb}{0.090, 0.745, 0.812}

%\def\argmax{\mathop{\rm argmax}}
%\def\argmin{\mathop{\rm argmin}}

\newcommand{\meas}[1]{$\text{#1}$}

\newcommand{\mybox}[2]{
\dashbox{\hbox{\hspace{0mm} \vbox{ \vspace{-1mm}
\textbf{#1:}

#2} \hspace{0mm} }}}

\newcommand{\dd}[2]{\frac{d#1}{d#2}}
\newcommand{\pp}[2]{\frac{\partial#1}{\partial#2}}

\newcommand{\ddx}{\frac{d}{dx}}
\newcommand{\ddy}{\frac{d}{dy}}
\newcommand{\ddn}{\frac{d}{dn}}
\newcommand{\ddt}{\frac{d}{dt}}

% ceil floor
\usepackage{mathtools} % needed for DeclarePairedDelimiter
\DeclarePairedDelimiter\ceil{\lceil}{\rceil}
\DeclarePairedDelimiter\floor{\lfloor}{\rfloor}

% P:() B:[] V:|| C:{} A:<>
\DeclarePairedDelimiter\BigP{\Big(}{\Big)}
\DeclarePairedDelimiter\BiggP{\Bigg(}{\Bigg)}
\DeclarePairedDelimiter\BigB{\Big[}{\Big]}
\DeclarePairedDelimiter\BiggB{\Bigg[}{\Bigg]}
\DeclarePairedDelimiter\BigV{\Big|}{\Big|}
\DeclarePairedDelimiter\BiggV{\Bigg|}{\Bigg|}
\DeclarePairedDelimiter\BigC{\Big\{}{\Big\}}
\DeclarePairedDelimiter\BiggC{\Bigg\{}{\Bigg\}}
\DeclarePairedDelimiter\BigA{\Big\langle}{\Big\rangle}
\DeclarePairedDelimiter\BiggA{\Bigg\langle}{\Bigg\rangle}

\newcommand{\sigmasq}{\sigma^2}
\newcommand{\chisq}{\chi^2}
\newcommand{\hint}[1]{$[$\textit{Hint:} #1$]$}

% Resize display math
% https://tex.stackexchange.com/questions/95821/how-to-scale-or-resize-equation-with-eqno-in-latex#378327
\newcommand\scalemath[2]{\scalebox{#1}{\mbox{\ensuremath{\displaystyle #2}}}}

% From 18.002
\DeclareMathOperator*{\argmax}{\arg\!\max}
\DeclareMathOperator*{\argmin}{\arg\!\min}

\newcommand{\myline}{\par\noindent\rule{\textwidth}{0.4pt}}
\newcommand{\mydash}{\par\noindent\dotfill}

\newcommand{\reals}{\mathbb{R}}
\newcommand{\naturals}{\mathbb{N}}
\newcommand{\integers}{\mathbb{Z}}
\newcommand{\rationals}{\mathbb{Q}}
\newcommand{\integrable}{\mathscr{R}}

\newcommand{\amp}{&&&&&&&&&&&&}
\newcommand{\st}{\text{ s.t. }}
\newcommand{\qst}{\quad\st\quad}
\newcommand{\qqst}{\qquad\st\qquad}
\newcommand{\qrightarrow}{\quad\rightarrow\quad}
\newcommand{\qqrightarrow}{\qquad\rightarrow\qquad}
\newcommand{\qRightarrow}{\quad\Rightarrow\quad}
\newcommand{\qqRightarrow}{\qquad\Rightarrow\qquad}
\newcommand{\qleftrightarrow}{\quad\leftrightarrow\quad}
\newcommand{\qqleftrightarrow}{\qquad\leftrightarrow\qquad}
\newcommand{\qLeftrightarrow}{\quad\Leftrightarrow\quad}
\newcommand{\qqLeftrightarrow}{\qquad\Leftrightarrow\qquad}
\newcommand{\qcolon}{\quad:\quad}
\newcommand{\qqcolon}{\qquad:\qquad}

% RA-L math symbols
\DeclarePairedDelimiter\tup{\langle}{\rangle}

\newcommand{\I}{\mathcal{I}}
\newcommand{\F}{\mathcal{F}}
\newcommand{\R}{\mathcal{R}}
\renewcommand{\P}{\mathcal{P}}
\renewcommand{\L}{\ell}
\newcommand{\loss}{\L}

\newcommand{\SE}[1]{SE(#1)}
\newcommand{\SO}[1]{SO(#1)}

\newcommand{\anglefull}{[-180^\circ,180^\circ]}
\newcommand{\anglehalf}{[-90^\circ,90^\circ]}

\renewcommand{\a}{A}
\renewcommand{\b}{B}
\newcommand{\w}{W}
\newcommand{\ab}{^{\a}_{\b}}

\newcommand{\A}{A}
\newcommand{\B}{B}
\newcommand{\Ai}{{\A_i}}
\newcommand{\Bj}{{\B_j}}
\newcommand{\AB}{^{\A}_{\B}}
\newcommand{\AiBj}{^\Ai_\Bj}
\newcommand{\wA}{^{\w}_{\A}}
\newcommand{\wB}{^{\w}_{\B}}
\newcommand{\wR}{^{\w}_{R}}

\newcommand{\NR}{N_R}
\newcommand{\NA}{{N_\A}}
\newcommand{\NB}{{N_\B}}

\newcommand{\p}{\mathbf{p}}
\newcommand{\ph}{\breve{\p}}

\newcommand{\T}{\mathbf{T}}
\newcommand{\Tab}{\T\ab}
\newcommand{\TAB}{\T\AB}
\newcommand{\TAiBj}{\T\AiBj}

\newcommand{\Rot}{\mathbf{R}}
\newcommand{\Rotx}{\Rot_x}
\newcommand{\Roty}{\Rot_y}
\newcommand{\Rotz}{\Rot_z}
\newcommand{\Rotab}{\Rot\ab}
\newcommand{\RotAB}{\Rot\AB}

\newcommand{\x}{x}
\newcommand{\xab}{\x\ab}
\newcommand{\xAB}{\x\AB}

\newcommand{\y}{y}
\newcommand{\yab}{\y\ab}
\newcommand{\yAB}{\y\AB}

\newcommand{\z}{z}
\newcommand{\zab}{\z\ab}
\newcommand{\zAB}{\z\AB}

\newcommand{\roll}{\alpha}
\newcommand{\rollab}{\roll\ab}
\newcommand{\rollAB}{\roll\AB}

\newcommand{\pitch}{\beta}
\newcommand{\pitchab}{\pitch\ab}
\newcommand{\pitchAB}{\pitch\AB}

\newcommand{\yaw}{\gamma}
\newcommand{\yawab}{\yaw\ab}
\newcommand{\yawAB}{\yaw\AB}

\newcommand{\trans}{\mathbf{t}}
\newcommand{\transab}{\trans\ab}
\newcommand{\transAB}{\trans\AB}

\newcommand{\envel}{\bm{\kappa}}
\newcommand{\envelc}{\hat{\envel}}
\newcommand{\envelt}{\check{\envel}}
\newcommand{\envelm}{\tilde{\envel}}

\newcommand{\zc}{\hat{\z}}
\newcommand{\rollc}{\hat{\roll}}
\newcommand{\pitchc}{\hat{\pitch}}

\newcommand{\zt}{\check{\z}}
\newcommand{\rollt}{\check{\roll}}
\newcommand{\pitcht}{\check{\pitch}}

\newcommand{\zm}{\tilde{\z}}
\newcommand{\rollm}{\tilde{\roll}}
\newcommand{\pitchm}{\tilde{\pitch}}

\renewcommand{\d}{d}
\newcommand{\dm}{\tilde{\d}}
\newcommand{\dn}{\bar{\d}}

\newcommand{\dv}{\mathbf{\d}}
\newcommand{\dvm}{\tilde{\dv}}
\newcommand{\dvn}{\bar{\dv}}

\newcommand{\e}{e}

\newcommand{\sumi}{\sum_{i = 1}^{\NA}}
\newcommand{\sumj}{\sum_{j = 1}^{\NB}}

\newcommand{\bs}{\vspace{1mm} \noindent}
\newcommand{\bns}{\noindent}

\newcommand{\edit}[1]{\textcolor{blue}{#1}}

% https://tex.stackexchange.com/a/9372
\newcommand{\mytexttilde}{\raisebox{0.5ex}{\texttildelow}}

\newcommand{\frameto}[2]{^{#1}_{#2}}
\newcommand{\var}[1]{\texttt{var}\left[#1\right]}
\newcommand{\std}[1]{\texttt{std}\left[#1\right]}


\title{Table II Explanation}
\author{Andrew Fishberg}
\date{}

\begin{document}

\maketitle

The values in Table II are calculated similarly to Dilution of Precision (DOP), a common uncertainty metric in GPS \cite{langley1999dilution}. Specifically, we are computing Position DOP (PDOP), the norm of the $x$, $y$, $z$ covariances (i.e., $\sigma_{xx}$, $\sigma_{yy}$, $\sigma_{zz}$), after being propagated through the nonlinear geometry linearized with respect to position and then evaluated at some relative pose $\TAB$.

This document breaks down this computation and shows how these values are calculated in the accompanying Jupyter Notebook.

\section{Problem Formulation:}

Consider two agents, $A$ and $B$, in world frame $W$ with true state vectors:
$$\mathbf{x}\frameto{W}{A} = \begin{bmatrix} x\frameto{W}{A} & y\frameto{W}{A} & z\frameto{W}{A} & \roll\frameto{W}{A} & \pitch\frameto{W}{A} & \yaw\frameto{W}{A} \end{bmatrix}^\top = \mathbf{T}\frameto{W}{A}$$
$$\mathbf{x}\frameto{W}{B} = \begin{bmatrix} x\frameto{W}{B} & y\frameto{W}{B} & z\frameto{W}{B} & \roll\frameto{W}{B} & \pitch\frameto{W}{B} & \yaw\frameto{W}{B} \end{bmatrix}^\top = \mathbf{T}\frameto{W}{B}$$
where $\tup{x,y,z}$ refers to position and $\tup{\alpha, \beta, \gamma}$ refers to roll/pitch/yaw, superscript represents the frame, and subscript represents the target. While in general $\mathbf{x} \leftrightarrow \mathbf{T}$, $\mathbf{x}$ refers to a column state vector while $\mathbf{T}$ refers to $\reals^{4\times 4}$ transformation matrix. Additionally, let $\mathbf{T}(\cdot)$ refer to the function that constructs a transformation matrix. Specifically, the following notations and shorthands are equivalent:
%$$\mathbf{T}\frameto{A}{B} = \mathbf{T}(\mathbf{x}\frameto{A}{B}) = \mathbf{T}(x\frameto{A}{B}, y\frameto{A}{B}, z\frameto{A}{B}, \roll\frameto{A}{B}, \pitch\frameto{A}{B}, \yaw\frameto{A}{B}) = \left(\mathbf{T}\frameto{W}{A}\right)^{-1} \mathbf{T}\frameto{W}{B}$$
\begin{align*}
\mathbf{T}(\mathbf{x}) &= \mathbf{T}(x, y, z, \roll, \pitch, \yaw) \\
\mathbf{T}(x, y, z) &= \mathbf{T}(x, y, z, 0, 0, 0) \\
\mathbf{T}(x, y, z, \yaw) &= \mathbf{T}(x, y, z, 0, 0, \yaw)
\end{align*}
Let the relative state from agent $A$ to agent $B$ be defined as:
$$\mathbf{x} = \mathbf{x}\frameto{A}{B} = \begin{bmatrix} x\frameto{A}{B} & y\frameto{A}{B} & z\frameto{A}{B} & \roll\frameto{A}{B} & \pitch\frameto{A}{B} & \yaw\frameto{A}{B} \end{bmatrix}^\top = \left(\mathbf{T}\frameto{W}{A}\right)^{-1} \mathbf{T}\frameto{W}{B} = \mathbf{T}\frameto{A}{B}$$
Relative coordinate frames are assumed to be oriented with respect to downward gravity.

\noindent Here, we have:
\begin{itemize}
\item $\hat{\mathbf{x}}$ are the estimated state parameters
\item $\mathbf{z}$ are the current measurements
\end{itemize}
where:
\begin{itemize}
\item $\hat{\mathbf{x}} \in \reals^3 \implies \hat{\mathbf{x}} = \begin{bmatrix}\hat{x}\frameto{A}{B} & \hat{y}\frameto{A}{B} & \hat{z}\frameto{A}{B}\end{bmatrix}$
\item $\mathbf{z} \in \reals^{37}$ has measurement model:
$$\mathbf{h}(\mathbf{x}\frameto{A}{B}) = \begin{bmatrix} \mathbf{d} & z\frameto{A}{B} \end{bmatrix}^\top$$
where $z\frameto{A}{B} \in \reals$ is the relative altitude and $\mathbf{d}$ is the vector of true UWB range. In this case, there are $\mathbf{d} \in \reals^{36}$ since agent $A$ and $B$ have $N_A = 6$ and $N_B = 6$ antennas respectively. In general, $\mathbf{d}$ has the form:
$$\mathbf{d} = \begin{bmatrix} d_{11} & d_{12} & & \dots & d_{N_A N_B}\end{bmatrix}^\top$$
with each element defined as:
$$d_{ij} = d_{ij}(\mathbf{x}\frameto{A}{B}) = ||\mathbf{T}\frameto{A}{B}\mathbf{p}\frameto{B}{j} - \mathbf{p}\frameto{A}{i}||_2$$
where $\mathbf{p}\frameto{A}{i}$ and $\mathbf{p}\frameto{B}{j}$ is the homogeneous coordinate of agent $A$'s $i$th and $B$'s $j$th antenna in $A$'s and $B$'s frame respectively.
\end{itemize}
With this measurement model, we can convince ourselves the system is fully observable. Specifically, the direct inclusion of $z\frameto{A}{B}$ into the measurement model resolves the usual $\pm z$ ambiguity stemming from antennas only spanning their relative $xy$-plane.

We note that while $z\frameto{A}{B}$ is not directly measurable, since both local $z$-axes are aligned with gravity down, $z\frameto{A}{B}$ can be calculated as the difference of two realizable measurements:
$$z\frameto{A}{B} = z\frameto{W}{B} - z\frameto{W}{A}$$
In practice, while we can always locally measure $z\frameto{W}{A}$, we do not want to transmit $z\frameto{W}{B}$. This will need to be considered in our error analysis.

\clearpage
\section{Computing PDOP:}

Consider following formulas from \cite{langley1999dilution}:
$$\Sigma_x = (A^\top \Sigma_z^{-1} A)^{-1}$$
in general, which can be simplified to:
$$\Sigma_x = (A^\top A)^{-1} \sigma^2$$
when covariances are consistent across all measurements and independent. Here, we have:
\begin{itemize}
    \item $A$ is the linearized measurement model about some point $\mathbf{x}_0$
    \item $\Sigma_z$ is the covariance matrix of the measurements $\mathbf{z}$
    \item $\Sigma_x$ is the covariance matrix of the state variables $\mathbf{x}$ after being propagated through the linearized measurement model $A$
\end{itemize}

When calculating DOP, the diagonal values of $(A^\top A)^{-1}$ are \texttt{XDOP}, \texttt{YDOP}, \texttt{ZDOP} values (i.e., the variance amplification resulting from the linearized geometry in various state dimensions).

In general since $d_{ij}$ is a UWB measurement and $z\frameto{A}{B}$ is some calculated combination of a low variance LiDAR measurement and/or high variance assumption, we have:
$$\var{d_{ij}} \neq \var{z\frameto{A}{B}}$$
forcing us to use the former equation to calculate $\Sigma_x$. 

In the paper (Figure 2a), we experimentally measured our UWB measurement variance to be:
$$\var{d_{ij}} = 0.24\text{m}$$
For the purpose of this error analysis, we will assume the measurement error is modeled by a zero mean Gaussian.

Depending on our experimental setup, $z\frameto{W}{A}$ and $z\frameto{W}{B}$ are either measured by downward facing LiDAR (with a reported operating tolerance of $\pm 4\text{cm}$ in our operating range) or assumed by some pre-communicated (but locally monitored) global altitude tolerance (i.e., $\bar{z}\frameto{W}{A} \pm \tilde{z}\frameto{W}{A}$ and $\bar{z}\frameto{W}{B} \pm \tilde{z}\frameto{W}{B}$ respectively where $\bar{z}$ is the commanded altitude and $\tilde{z}$ is the altitude tolerance).

\clearpage
\section{Jupyter Notebook:}

Using the $\hat{\mathbf{x}} \in \reals^3$ and $\mathbf{z} \in \reals^{37}$ as defined above, we perform the above error analysis with different values for $\var{z\frameto{A}{B}}$ (selected based on current communication/sensor assumptions and encoded in $\Sigma_z$) and different measurement model linearization points (select based on current position and encoded in $A$) and calculate the corresponding PDOPs. This implies $A$ is of the form $\reals^{37 \times 3}$, making this a cumbersome computation to do by hand. Thus, we make the Jupyter Notebook used to compute Table II (and other relevant facts) available here:
\begin{itemize}
\item \url{https://github.com/mit-acl/murp-datasets}
\end{itemize}

\bibliographystyle{unsrt}
\bibliography{ref.bib}

\end{document}